\documentclass{article}

\usepackage[utf8]{inputenc}
\usepackage{unicode-helper}
\usepackage{amsmath}
\usepackage{amssymb}
\usepackage{amsfonts}
\usepackage{microtype}
\usepackage[ngerman]{babel}
\usepackage{amsthm}
\usepackage{mathtools}
\usepackage{braket}
\usepackage{csquotes}
\usepackage[pdftex]{hyperref}
\usepackage{cleveref}

\newcommand{\titlevar}{Lichter im Dunkeln}
\newcommand{\authorvar}{J. Günthner}
\newcommand{\datevar}{Sommer 25}
\title{\titlevar}
\author{\authorvar}
\date{\datevar}
\hypersetup{
	pdftitle=\titlevar,
	pdfauthor=\authorvar,
	pdfcreationdate=\datevar,
}
\setlength{\parindent}{0pt}

\begin{document}
	\maketitle

	Schwere Depression. Stand da. Gebäude ziehen an mir vorbei. Lichter im Dunkeln. Eins. Dann das nächste. Und noch eins. Naja. Ich hole das Zettelchen raus. Schwere Depresion. Ha! Ha. Eine ganze Weile starre ich in das Dunkel und gedenke meiner Vergangenheit. Es war mal besser. Die Melancholie ist lämend.

	\medskip

	Eine leise Träne läuft meine Wange hinunter.

	\medskip

	Außer ihr, keine Regung in meinem Körper oder meinem Kopf.

	\medskip

	Ein Blick durch den Zug spiegelt meine Gefühle wieder. Völlig leer und allein. Nur Regen fehlt noch. Die Lichterlandschaft schimmert wie der Sternenhimmel auf dem Land.

	\medskip

	In den letzten paar Monaten bin ich komplett kollabiert. Ohne es wahrhaben zu wollen. Immer weniger schaffte ich selbst. Arbeiten, kochen, essen, schlafen, aufstehen. Dann bin ich zurück zu meinen Eltern gezogen, um nicht zu verhungern. Es fühlt sich alles zu leer an. Es ist schwer selbst die Dinge zu machen, für die ich normalerweise brenne. Ständig verziehe ich mich in die online Welten anderer. Tauche darin ein—nein, ich gehe darin unter. Ich will nicht mehr. Aber wie früher auch ist meine Angst vor dem Tod größer.

	\medskip

	Durch die Leere meiner Gefühle dringt nur eine einzelne Träne. Wo ich doch eigentlich fast gar nicht mehr weinen kann. Früher konnte ich diesem Gefühl noch freien Lauf lassen, aber irgendwann bin ich abgestumpft. So viel Trauer damals, heute nur noch das Gefühl von Taubheit. Als hätte man mir das Bein abgehackt nachdem man mich betäubt hätte. Den Schmerz spüre ich immernoch, aber nur ganz dumpf.

	\medskip

	Dabei lief es für ein ganzes Jahr richtig gut. Ich hatte einen Job gefunden, der mir richtig viel Spaß machte, ja eigentlich macht er mir immer noch Spaß. Nur kann ich mich auch kaum noch darauf konzentrieren. Und jetzt bin ich erstmal wieder arbeitslos. Etwas auf dem meine Mutter leider nur verständnislos rumhackt. Kein schöner Gedanke, dahin zurückzufahren. Aber unterwegs bin ich ja schon. Und zu meiner Wohnung, obwohl nicht gekündigt, könnte ich jetzt auch nicht zurück. Ich fürchte, dass ich es nicht mehr schaffen würde, die zu verlassen.

	\medskip

	Ich würde schreien, wenn ich es noch könnte. Vor wenigen Jahren war das noch, das ich diesem Gefühl so freien Lauf lassen konnte. Tatsächlich hat es jetzt angefangen zu regnen. Wassertropfen fliegen in die Scheibe wie Spatzen oder Tauben. Ein ständiges Hämmern. Und der Glanz der Stadt schimmert durch die Tropfen. Verzerrt. Ich blicke hinunter auf meinen Unterarm. Auf die Narben, die dort mal waren. Ich kann sie immernoch sehen. Wie ich damals im Klassenzimmer saß und auch auf sie hinunter blickte auf meinen Unterarm.

	\medskip

	Manche Dinge würde ich so gern vergessen. Es fühlt sich so an, als würden dann und nur dann auch diese Gefühle verschwinden. Dieser Schmerz verschwinden. Aber all das ist eingebrannt, zwischen all den Umarmungen, dem Lachen, den Freunden. Den Freunden. Auch das ist mir abhandengekommen. Es ist schwer, sich zu verabreden, wenn man es auch nicht aus dem Haus schafft. Wenigstens bleibt mir noch meine Therapeutin, habe ich es doch geschafft dorthin, zumindest heute.

	\medskip

	Aber auch das schaffe ich viel zu oft nicht. Auch schon vor meinem langsamen Kollaps. Nicht so häufig wie jetzt, dennoch. Vielleicht habe ich nicht wirklich gemerkt, dass mich meine Arbeit immer weniger motiviert hat. Mir immer weniger Last von den Schultern genommen hat. Eher mir eine auferlegt hat. War es doch am Anfang noch ein unglaublich schönes Gefühl, war es mit den Monaten immer öfter auch einfach anstrengend. Vielleicht kann das auf lange Zeit für mich auch einfach nicht funktinieren, wenn ich einfach immer weiterarbeite, angespornt durch meinen Erfolg—vor allem durch meinen Erfolg gemessen, an meinem früheren——wirkliche Freizeit habe ich mir fast nie gegönnt.

	\medskip

	Und jetzt das. Meine Oma. Wie eine Mutter für mich. Tot. 

	\medskip

	\enquote{jetzt}.

	\medskip

	Das war vor den paar Monaten. Auch vor ihremm Tod war sie schon fast gar nicht mehr da. Sie hatte Demenz. Und ihr körperlicher Zustand ähnelte schon einem Skelett. Früher war sie noch ein Licht in einem Fenster, wie die hier alle. Und dann eines morgens—oder abends—oder mittags—da ging das aus. Aber niemand hatte den Lichtschalter getätigt, nein. Eher ist die Birne durchgebrannt. Hat den Geist aufgegeben. Und der ist jetzt woanders.

	\medskip

	Ich vermisse sie so.

	\medskip

	Doch ich kann nichts tun. Ich konnte nichts tun. Und ich weiß nicht wie ich damit umgehen soll. Es scheint unmöglich. Unüberwindbar.
\end{document} 
